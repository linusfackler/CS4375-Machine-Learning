% Options for packages loaded elsewhere
\PassOptionsToPackage{unicode}{hyperref}
\PassOptionsToPackage{hyphens}{url}
%
\documentclass[
]{article}
\usepackage{amsmath,amssymb}
\usepackage{lmodern}
\usepackage{iftex}
\ifPDFTeX
  \usepackage[T1]{fontenc}
  \usepackage[utf8]{inputenc}
  \usepackage{textcomp} % provide euro and other symbols
\else % if luatex or xetex
  \usepackage{unicode-math}
  \defaultfontfeatures{Scale=MatchLowercase}
  \defaultfontfeatures[\rmfamily]{Ligatures=TeX,Scale=1}
\fi
% Use upquote if available, for straight quotes in verbatim environments
\IfFileExists{upquote.sty}{\usepackage{upquote}}{}
\IfFileExists{microtype.sty}{% use microtype if available
  \usepackage[]{microtype}
  \UseMicrotypeSet[protrusion]{basicmath} % disable protrusion for tt fonts
}{}
\makeatletter
\@ifundefined{KOMAClassName}{% if non-KOMA class
  \IfFileExists{parskip.sty}{%
    \usepackage{parskip}
  }{% else
    \setlength{\parindent}{0pt}
    \setlength{\parskip}{6pt plus 2pt minus 1pt}}
}{% if KOMA class
  \KOMAoptions{parskip=half}}
\makeatother
\usepackage{xcolor}
\usepackage[margin=1in]{geometry}
\usepackage{color}
\usepackage{fancyvrb}
\newcommand{\VerbBar}{|}
\newcommand{\VERB}{\Verb[commandchars=\\\{\}]}
\DefineVerbatimEnvironment{Highlighting}{Verbatim}{commandchars=\\\{\}}
% Add ',fontsize=\small' for more characters per line
\usepackage{framed}
\definecolor{shadecolor}{RGB}{248,248,248}
\newenvironment{Shaded}{\begin{snugshade}}{\end{snugshade}}
\newcommand{\AlertTok}[1]{\textcolor[rgb]{0.94,0.16,0.16}{#1}}
\newcommand{\AnnotationTok}[1]{\textcolor[rgb]{0.56,0.35,0.01}{\textbf{\textit{#1}}}}
\newcommand{\AttributeTok}[1]{\textcolor[rgb]{0.77,0.63,0.00}{#1}}
\newcommand{\BaseNTok}[1]{\textcolor[rgb]{0.00,0.00,0.81}{#1}}
\newcommand{\BuiltInTok}[1]{#1}
\newcommand{\CharTok}[1]{\textcolor[rgb]{0.31,0.60,0.02}{#1}}
\newcommand{\CommentTok}[1]{\textcolor[rgb]{0.56,0.35,0.01}{\textit{#1}}}
\newcommand{\CommentVarTok}[1]{\textcolor[rgb]{0.56,0.35,0.01}{\textbf{\textit{#1}}}}
\newcommand{\ConstantTok}[1]{\textcolor[rgb]{0.00,0.00,0.00}{#1}}
\newcommand{\ControlFlowTok}[1]{\textcolor[rgb]{0.13,0.29,0.53}{\textbf{#1}}}
\newcommand{\DataTypeTok}[1]{\textcolor[rgb]{0.13,0.29,0.53}{#1}}
\newcommand{\DecValTok}[1]{\textcolor[rgb]{0.00,0.00,0.81}{#1}}
\newcommand{\DocumentationTok}[1]{\textcolor[rgb]{0.56,0.35,0.01}{\textbf{\textit{#1}}}}
\newcommand{\ErrorTok}[1]{\textcolor[rgb]{0.64,0.00,0.00}{\textbf{#1}}}
\newcommand{\ExtensionTok}[1]{#1}
\newcommand{\FloatTok}[1]{\textcolor[rgb]{0.00,0.00,0.81}{#1}}
\newcommand{\FunctionTok}[1]{\textcolor[rgb]{0.00,0.00,0.00}{#1}}
\newcommand{\ImportTok}[1]{#1}
\newcommand{\InformationTok}[1]{\textcolor[rgb]{0.56,0.35,0.01}{\textbf{\textit{#1}}}}
\newcommand{\KeywordTok}[1]{\textcolor[rgb]{0.13,0.29,0.53}{\textbf{#1}}}
\newcommand{\NormalTok}[1]{#1}
\newcommand{\OperatorTok}[1]{\textcolor[rgb]{0.81,0.36,0.00}{\textbf{#1}}}
\newcommand{\OtherTok}[1]{\textcolor[rgb]{0.56,0.35,0.01}{#1}}
\newcommand{\PreprocessorTok}[1]{\textcolor[rgb]{0.56,0.35,0.01}{\textit{#1}}}
\newcommand{\RegionMarkerTok}[1]{#1}
\newcommand{\SpecialCharTok}[1]{\textcolor[rgb]{0.00,0.00,0.00}{#1}}
\newcommand{\SpecialStringTok}[1]{\textcolor[rgb]{0.31,0.60,0.02}{#1}}
\newcommand{\StringTok}[1]{\textcolor[rgb]{0.31,0.60,0.02}{#1}}
\newcommand{\VariableTok}[1]{\textcolor[rgb]{0.00,0.00,0.00}{#1}}
\newcommand{\VerbatimStringTok}[1]{\textcolor[rgb]{0.31,0.60,0.02}{#1}}
\newcommand{\WarningTok}[1]{\textcolor[rgb]{0.56,0.35,0.01}{\textbf{\textit{#1}}}}
\usepackage{graphicx}
\makeatletter
\def\maxwidth{\ifdim\Gin@nat@width>\linewidth\linewidth\else\Gin@nat@width\fi}
\def\maxheight{\ifdim\Gin@nat@height>\textheight\textheight\else\Gin@nat@height\fi}
\makeatother
% Scale images if necessary, so that they will not overflow the page
% margins by default, and it is still possible to overwrite the defaults
% using explicit options in \includegraphics[width, height, ...]{}
\setkeys{Gin}{width=\maxwidth,height=\maxheight,keepaspectratio}
% Set default figure placement to htbp
\makeatletter
\def\fps@figure{htbp}
\makeatother
\setlength{\emergencystretch}{3em} % prevent overfull lines
\providecommand{\tightlist}{%
  \setlength{\itemsep}{0pt}\setlength{\parskip}{0pt}}
\setcounter{secnumdepth}{-\maxdimen} % remove section numbering
\ifLuaTeX
  \usepackage{selnolig}  % disable illegal ligatures
\fi
\IfFileExists{bookmark.sty}{\usepackage{bookmark}}{\usepackage{hyperref}}
\IfFileExists{xurl.sty}{\usepackage{xurl}}{} % add URL line breaks if available
\urlstyle{same} % disable monospaced font for URLs
\hypersetup{
  pdftitle={Regression},
  pdfauthor={Linus Fackler, Justin Hardy, Fernando Colman, Isabelle Villegas},
  hidelinks,
  pdfcreator={LaTeX via pandoc}}

\title{Regression}
\author{Linus Fackler, Justin Hardy, Fernando Colman, Isabelle Villegas}
\date{}

\begin{document}
\maketitle

This data set contains information of over 10000 different plane rides
in India in 2019, containing the price, duration, number of stops, and
other information. The set can be found here:
\url{https://www.kaggle.com/datasets/ibrahimelsayed182/plane-ticket-price}

\hypertarget{load-the-data}{%
\subsubsection{Load the data}\label{load-the-data}}

\begin{Shaded}
\begin{Highlighting}[]
\NormalTok{df }\OtherTok{\textless{}{-}} \FunctionTok{read.csv}\NormalTok{(}\StringTok{"planetickets.csv"}\NormalTok{, }\AttributeTok{header =} \ConstantTok{TRUE}\NormalTok{)}
\FunctionTok{str}\NormalTok{(df)}
\end{Highlighting}
\end{Shaded}

\begin{verbatim}
## 'data.frame':    10683 obs. of  11 variables:
##  $ Airline        : chr  "IndiGo" "Air India" "Jet Airways" "IndiGo" ...
##  $ Date_of_Journey: chr  "24/03/2019" "1/05/2019" "9/06/2019" "12/05/2019" ...
##  $ Source         : chr  "Banglore" "Kolkata" "Delhi" "Kolkata" ...
##  $ Destination    : chr  "New Delhi" "Banglore" "Cochin" "Banglore" ...
##  $ Route          : chr  "BLR ? DEL" "CCU ? IXR ? BBI ? BLR" "DEL ? LKO ? BOM ? COK" "CCU ? NAG ? BLR" ...
##  $ Dep_Time       : chr  "22:20" "05:50" "09:25" "18:05" ...
##  $ Arrival_Time   : chr  "01:10 22 Mar" "13:15" "04:25 10 Jun" "23:30" ...
##  $ Duration       : chr  "2h 50m" "7h 25m" "19h" "5h 25m" ...
##  $ Total_Stops    : chr  "non-stop" "2 stops" "2 stops" "1 stop" ...
##  $ Additional_Info: chr  "No info" "No info" "No info" "No info" ...
##  $ Price          : int  3897 7662 13882 6218 13302 3873 11087 22270 11087 8625 ...
\end{verbatim}

\hypertarget{data-cleaning}{%
\subsubsection{Data cleaning}\label{data-cleaning}}

First, we have to clean the data, since a lot of columns contain strings
instead of integers or floats.

We throw out some of the columns that don't provide any useful
information, such as ``additional info''.

\begin{Shaded}
\begin{Highlighting}[]
\NormalTok{df }\OtherTok{\textless{}{-}}\NormalTok{ df[,}\FunctionTok{c}\NormalTok{(}\DecValTok{1}\NormalTok{,}\DecValTok{3}\NormalTok{,}\DecValTok{4}\NormalTok{,}\DecValTok{6}\NormalTok{,}\DecValTok{8}\NormalTok{,}\DecValTok{9}\NormalTok{,}\DecValTok{11}\NormalTok{)]}
\end{Highlighting}
\end{Shaded}

We check how many NA rows there are.

\begin{Shaded}
\begin{Highlighting}[]
\FunctionTok{sapply}\NormalTok{(df, }\ControlFlowTok{function}\NormalTok{(x) }\FunctionTok{sum}\NormalTok{(}\FunctionTok{is.na}\NormalTok{(x)}\SpecialCharTok{==}\ConstantTok{TRUE}\NormalTok{))}
\end{Highlighting}
\end{Shaded}

\begin{verbatim}
##     Airline      Source Destination    Dep_Time    Duration Total_Stops 
##           0           0           0           0           0           0 
##       Price 
##           0
\end{verbatim}

This data set contains no rows with missing values, so we don't have to
make any changes.

Now, we check how many unique values the column Total\_Stops has, before
we change it from a String to an Integer.

\begin{Shaded}
\begin{Highlighting}[]
\FunctionTok{length}\NormalTok{(}\FunctionTok{unique}\NormalTok{(df}\SpecialCharTok{$}\NormalTok{Total\_Stops))}
\end{Highlighting}
\end{Shaded}

\begin{verbatim}
## [1] 6
\end{verbatim}

\begin{Shaded}
\begin{Highlighting}[]
\FunctionTok{unique}\NormalTok{(df}\SpecialCharTok{$}\NormalTok{Total\_Stops)}
\end{Highlighting}
\end{Shaded}

\begin{verbatim}
## [1] "non-stop" "2 stops"  "1 stop"   "3 stops"  ""         "4 stops"
\end{verbatim}

\begin{Shaded}
\begin{Highlighting}[]
\FunctionTok{sum}\NormalTok{(df}\SpecialCharTok{$}\NormalTok{Total\_Stops }\SpecialCharTok{==} \StringTok{""}\NormalTok{)}
\end{Highlighting}
\end{Shaded}

\begin{verbatim}
## [1] 1
\end{verbatim}

We see that it has 6 unique values, one of them is just the empty
string. There is only 1 row with this empty value, so we delete this
row.

\begin{Shaded}
\begin{Highlighting}[]
\NormalTok{df }\OtherTok{\textless{}{-}}\NormalTok{ df[}\SpecialCharTok{!}\NormalTok{(df}\SpecialCharTok{$}\NormalTok{Total\_Stops }\SpecialCharTok{==} \StringTok{""}\NormalTok{),]}
\FunctionTok{unique}\NormalTok{(df}\SpecialCharTok{$}\NormalTok{Total\_Stops)}
\end{Highlighting}
\end{Shaded}

\begin{verbatim}
## [1] "non-stop" "2 stops"  "1 stop"   "3 stops"  "4 stops"
\end{verbatim}

We change these to usable integer values and then change the type of the
column to numeric.

\begin{Shaded}
\begin{Highlighting}[]
\NormalTok{df}\SpecialCharTok{$}\NormalTok{Total\_Stops[df}\SpecialCharTok{$}\NormalTok{Total\_Stops }\SpecialCharTok{==} \StringTok{"non{-}stop"}\NormalTok{] }\OtherTok{\textless{}{-}} \DecValTok{0}
\NormalTok{df}\SpecialCharTok{$}\NormalTok{Total\_Stops[df}\SpecialCharTok{$}\NormalTok{Total\_Stops }\SpecialCharTok{==} \StringTok{"1 stop"}\NormalTok{] }\OtherTok{\textless{}{-}} \DecValTok{1}
\NormalTok{df}\SpecialCharTok{$}\NormalTok{Total\_Stops[df}\SpecialCharTok{$}\NormalTok{Total\_Stops }\SpecialCharTok{==} \StringTok{"2 stops"}\NormalTok{] }\OtherTok{\textless{}{-}} \DecValTok{2}
\NormalTok{df}\SpecialCharTok{$}\NormalTok{Total\_Stops[df}\SpecialCharTok{$}\NormalTok{Total\_Stops }\SpecialCharTok{==} \StringTok{"3 stops"}\NormalTok{] }\OtherTok{\textless{}{-}} \DecValTok{3}
\NormalTok{df}\SpecialCharTok{$}\NormalTok{Total\_Stops[df}\SpecialCharTok{$}\NormalTok{Total\_Stops }\SpecialCharTok{==} \StringTok{"4 stops"}\NormalTok{] }\OtherTok{\textless{}{-}} \DecValTok{4}

\FunctionTok{unique}\NormalTok{(df}\SpecialCharTok{$}\NormalTok{Total\_Stops)}
\end{Highlighting}
\end{Shaded}

\begin{verbatim}
## [1] "0" "2" "1" "3" "4"
\end{verbatim}

\begin{Shaded}
\begin{Highlighting}[]
\NormalTok{df }\OtherTok{\textless{}{-}} \FunctionTok{transform}\NormalTok{(df, }\AttributeTok{Total\_Stops =} \FunctionTok{as.integer}\NormalTok{(Total\_Stops))}
\FunctionTok{str}\NormalTok{(df)}
\end{Highlighting}
\end{Shaded}

\begin{verbatim}
## 'data.frame':    10682 obs. of  7 variables:
##  $ Airline    : chr  "IndiGo" "Air India" "Jet Airways" "IndiGo" ...
##  $ Source     : chr  "Banglore" "Kolkata" "Delhi" "Kolkata" ...
##  $ Destination: chr  "New Delhi" "Banglore" "Cochin" "Banglore" ...
##  $ Dep_Time   : chr  "22:20" "05:50" "09:25" "18:05" ...
##  $ Duration   : chr  "2h 50m" "7h 25m" "19h" "5h 25m" ...
##  $ Total_Stops: int  0 2 2 1 1 0 1 1 1 1 ...
##  $ Price      : int  3897 7662 13882 6218 13302 3873 11087 22270 11087 8625 ...
\end{verbatim}

Now, for the column ``Duration'', we will cut off everything after the
`h', so we only keep the hours of the flight.

\begin{Shaded}
\begin{Highlighting}[]
\NormalTok{df}\SpecialCharTok{$}\NormalTok{Duration }\OtherTok{\textless{}{-}} \FunctionTok{gsub}\NormalTok{(}\StringTok{"h.*"}\NormalTok{,}\StringTok{""}\NormalTok{, df}\SpecialCharTok{$}\NormalTok{Duration)}
\NormalTok{df}\SpecialCharTok{$}\NormalTok{Duration[}\DecValTok{1}\SpecialCharTok{:}\DecValTok{10}\NormalTok{]}
\end{Highlighting}
\end{Shaded}

\begin{verbatim}
##  [1] "2"  "7"  "19" "5"  "4"  "2"  "15" "21" "25" "7"
\end{verbatim}

Then, we change the type also to numeric.

\begin{Shaded}
\begin{Highlighting}[]
\NormalTok{df }\OtherTok{\textless{}{-}} \FunctionTok{transform}\NormalTok{(df, }\AttributeTok{Duration =} \FunctionTok{as.integer}\NormalTok{(Duration))}
\end{Highlighting}
\end{Shaded}

\begin{verbatim}
## Warning in eval(substitute(list(...)), `_data`, parent.frame()): NAs introduced
## by coercion
\end{verbatim}

\begin{Shaded}
\begin{Highlighting}[]
\FunctionTok{str}\NormalTok{(df)}
\end{Highlighting}
\end{Shaded}

\begin{verbatim}
## 'data.frame':    10682 obs. of  7 variables:
##  $ Airline    : chr  "IndiGo" "Air India" "Jet Airways" "IndiGo" ...
##  $ Source     : chr  "Banglore" "Kolkata" "Delhi" "Kolkata" ...
##  $ Destination: chr  "New Delhi" "Banglore" "Cochin" "Banglore" ...
##  $ Dep_Time   : chr  "22:20" "05:50" "09:25" "18:05" ...
##  $ Duration   : int  2 7 19 5 4 2 15 21 25 7 ...
##  $ Total_Stops: int  0 2 2 1 1 0 1 1 1 1 ...
##  $ Price      : int  3897 7662 13882 6218 13302 3873 11087 22270 11087 8625 ...
\end{verbatim}

Upon checking, there is somehow an NA value in our column Duration now.

\begin{Shaded}
\begin{Highlighting}[]
\FunctionTok{sapply}\NormalTok{(df, }\ControlFlowTok{function}\NormalTok{(x) }\FunctionTok{sum}\NormalTok{(}\FunctionTok{is.na}\NormalTok{(x)}\SpecialCharTok{==}\ConstantTok{TRUE}\NormalTok{))}
\end{Highlighting}
\end{Shaded}

\begin{verbatim}
##     Airline      Source Destination    Dep_Time    Duration Total_Stops 
##           0           0           0           0           1           0 
##       Price 
##           0
\end{verbatim}

We will simply delete this row, as 1 row does not matter.

\begin{Shaded}
\begin{Highlighting}[]
\NormalTok{df }\OtherTok{\textless{}{-}}\NormalTok{ df[}\SpecialCharTok{!}\NormalTok{(}\FunctionTok{is.na}\NormalTok{(df}\SpecialCharTok{$}\NormalTok{Duration)),]}
\FunctionTok{sapply}\NormalTok{(df, }\ControlFlowTok{function}\NormalTok{(x) }\FunctionTok{sum}\NormalTok{(}\FunctionTok{is.na}\NormalTok{(x)}\SpecialCharTok{==}\ConstantTok{TRUE}\NormalTok{))}
\end{Highlighting}
\end{Shaded}

\begin{verbatim}
##     Airline      Source Destination    Dep_Time    Duration Total_Stops 
##           0           0           0           0           0           0 
##       Price 
##           0
\end{verbatim}

We're done with duration now.

We do the same for the departure time. We will cut off the minutes and
just keep the hours.

\begin{Shaded}
\begin{Highlighting}[]
\NormalTok{df}\SpecialCharTok{$}\NormalTok{Dep\_Time }\OtherTok{\textless{}{-}} \FunctionTok{gsub}\NormalTok{(}\StringTok{":.*"}\NormalTok{,}\StringTok{""}\NormalTok{, df}\SpecialCharTok{$}\NormalTok{Dep\_Time)}
\NormalTok{df}\SpecialCharTok{$}\NormalTok{Dep\_Time[}\DecValTok{1}\SpecialCharTok{:}\DecValTok{10}\NormalTok{]}
\end{Highlighting}
\end{Shaded}

\begin{verbatim}
##  [1] "22" "05" "09" "18" "16" "09" "18" "08" "08" "11"
\end{verbatim}

Now, we will transform it to an integer value.

\begin{Shaded}
\begin{Highlighting}[]
\NormalTok{df }\OtherTok{\textless{}{-}} \FunctionTok{transform}\NormalTok{(df, }\AttributeTok{Dep\_Time =} \FunctionTok{as.integer}\NormalTok{(Dep\_Time))}
\FunctionTok{str}\NormalTok{(df)}
\end{Highlighting}
\end{Shaded}

\begin{verbatim}
## 'data.frame':    10681 obs. of  7 variables:
##  $ Airline    : chr  "IndiGo" "Air India" "Jet Airways" "IndiGo" ...
##  $ Source     : chr  "Banglore" "Kolkata" "Delhi" "Kolkata" ...
##  $ Destination: chr  "New Delhi" "Banglore" "Cochin" "Banglore" ...
##  $ Dep_Time   : int  22 5 9 18 16 9 18 8 8 11 ...
##  $ Duration   : int  2 7 19 5 4 2 15 21 25 7 ...
##  $ Total_Stops: int  0 2 2 1 1 0 1 1 1 1 ...
##  $ Price      : int  3897 7662 13882 6218 13302 3873 11087 22270 11087 8625 ...
\end{verbatim}

\hypertarget{train-and-test-sets}{%
\subsubsection{Train and Test sets}\label{train-and-test-sets}}

Divide into train and test sets

\begin{Shaded}
\begin{Highlighting}[]
\FunctionTok{set.seed}\NormalTok{(}\DecValTok{1234}\NormalTok{)}
\NormalTok{i }\OtherTok{\textless{}{-}} \FunctionTok{sample}\NormalTok{(}\DecValTok{1}\SpecialCharTok{:}\FunctionTok{nrow}\NormalTok{(df), }\FunctionTok{round}\NormalTok{(}\FunctionTok{nrow}\NormalTok{(df)}\SpecialCharTok{*}\FloatTok{0.8}\NormalTok{), }\AttributeTok{replace=}\ConstantTok{FALSE}\NormalTok{)}
\NormalTok{train }\OtherTok{\textless{}{-}}\NormalTok{ df[i, }\SpecialCharTok{{-}}\DecValTok{9}\NormalTok{]}
\NormalTok{test }\OtherTok{\textless{}{-}}\NormalTok{ df[}\SpecialCharTok{{-}}\NormalTok{i, }\SpecialCharTok{{-}}\DecValTok{9}\NormalTok{]}
\end{Highlighting}
\end{Shaded}

\hypertarget{data-exploration-of-training-data}{%
\subsubsection{Data Exploration of Training
Data}\label{data-exploration-of-training-data}}

First, we look at how flight durations and ticket prices look together
in a plot.

\begin{Shaded}
\begin{Highlighting}[]
\FunctionTok{plot}\NormalTok{(train}\SpecialCharTok{$}\NormalTok{Duration}\SpecialCharTok{\textasciitilde{}}\NormalTok{train}\SpecialCharTok{$}\NormalTok{Price, }\AttributeTok{xlab=}\StringTok{"Ticket Price"}\NormalTok{, }\AttributeTok{ylab=}\StringTok{"Flight Duration"}\NormalTok{)}
\FunctionTok{abline}\NormalTok{(}\FunctionTok{lm}\NormalTok{(train}\SpecialCharTok{$}\NormalTok{Duration}\SpecialCharTok{\textasciitilde{}}\NormalTok{train}\SpecialCharTok{$}\NormalTok{Price), }\AttributeTok{col=}\StringTok{"red"}\NormalTok{)}
\end{Highlighting}
\end{Shaded}

\includegraphics{Regression_files/figure-latex/unnamed-chunk-14-1.pdf}

This shows us the general trend of longer flights resulting in higher
ticket prices.

Now, we will see how

\begin{Shaded}
\begin{Highlighting}[]
\NormalTok{x }\OtherTok{\textless{}{-}}\NormalTok{ train}\SpecialCharTok{$}\NormalTok{Price[(train}\SpecialCharTok{$}\NormalTok{Price }\SpecialCharTok{\textless{}} \DecValTok{15000}\NormalTok{) }\SpecialCharTok{\&}\NormalTok{ (train}\SpecialCharTok{$}\NormalTok{Total\_Stops }\SpecialCharTok{==} \DecValTok{0}\NormalTok{)]}
\NormalTok{y }\OtherTok{\textless{}{-}}\NormalTok{ train}\SpecialCharTok{$}\NormalTok{Duration[(train}\SpecialCharTok{$}\NormalTok{Duration }\SpecialCharTok{\textless{}} \DecValTok{20}\NormalTok{) }\SpecialCharTok{\&}\NormalTok{ (train}\SpecialCharTok{$}\NormalTok{Total\_Stops }\SpecialCharTok{==} \DecValTok{0}\NormalTok{)]}
\FunctionTok{plot}\NormalTok{(y[}\DecValTok{1}\SpecialCharTok{:}\FunctionTok{min}\NormalTok{(}\FunctionTok{length}\NormalTok{(x),}\FunctionTok{length}\NormalTok{(y))]}\SpecialCharTok{\textasciitilde{}}\NormalTok{x[}\DecValTok{1}\SpecialCharTok{:}\FunctionTok{min}\NormalTok{(}\FunctionTok{length}\NormalTok{(x),}\FunctionTok{length}\NormalTok{(y))], }\AttributeTok{xlab=}\StringTok{"Ticket Price"}\NormalTok{, }\AttributeTok{ylab=}\StringTok{"Flight Duration"}\NormalTok{, }\AttributeTok{main=}\StringTok{"Flights with 0 stops"}\NormalTok{)}
\end{Highlighting}
\end{Shaded}

\includegraphics{Regression_files/figure-latex/unnamed-chunk-15-1.pdf}

\begin{Shaded}
\begin{Highlighting}[]
\NormalTok{counts }\OtherTok{\textless{}{-}} \FunctionTok{table}\NormalTok{(train}\SpecialCharTok{$}\NormalTok{Total\_Stops)}
\FunctionTok{barplot}\NormalTok{(counts)}
\end{Highlighting}
\end{Shaded}

\includegraphics{Regression_files/figure-latex/unnamed-chunk-16-1.pdf}

We see that there is no data where the number of stops is 4 in the
training portion of the data, which is why we will leave it out in the
next plot.

\begin{Shaded}
\begin{Highlighting}[]
\NormalTok{colors }\OtherTok{\textless{}{-}} \FunctionTok{c}\NormalTok{(}\StringTok{"\#2EFF00"}\NormalTok{, }\CommentTok{\#green for 0 stops}
            \StringTok{"\#FF0000"}\NormalTok{, }\CommentTok{\#red for 1 stop}
            \StringTok{"\#FFFB00"}\NormalTok{, }\CommentTok{\#yellow for 2 stops}
            \StringTok{"\#000CFF"} \CommentTok{\#blue for 3 stops}
\NormalTok{            )}

\NormalTok{groups }\OtherTok{\textless{}{-}} \FunctionTok{factor}\NormalTok{(train}\SpecialCharTok{$}\NormalTok{Total\_Stops)}
\FunctionTok{plot}\NormalTok{(train}\SpecialCharTok{$}\NormalTok{Duration}\SpecialCharTok{\textasciitilde{}}\NormalTok{train}\SpecialCharTok{$}\NormalTok{Price, }\AttributeTok{xlab=}\StringTok{"Ticket Price"}\NormalTok{, }\AttributeTok{ylab=}\StringTok{"Flight Duration"}\NormalTok{, }\AttributeTok{pch =} \DecValTok{19}\NormalTok{, }\AttributeTok{col=}\NormalTok{colors[groups])}
\FunctionTok{legend}\NormalTok{(}\StringTok{"topright"}\NormalTok{, }\AttributeTok{legend=}\FunctionTok{c}\NormalTok{(}\StringTok{"0 stops"}\NormalTok{, }\StringTok{"1 stop"}\NormalTok{, }\StringTok{"2 stops"}\NormalTok{, }\StringTok{"3 stops"}\NormalTok{), }\AttributeTok{pch =} \DecValTok{19}\NormalTok{, }\AttributeTok{col =}\NormalTok{ colors[}\FunctionTok{factor}\NormalTok{(}\FunctionTok{levels}\NormalTok{(groups))])}
\FunctionTok{abline}\NormalTok{(}\FunctionTok{lm}\NormalTok{(train}\SpecialCharTok{$}\NormalTok{Duration}\SpecialCharTok{\textasciitilde{}}\NormalTok{train}\SpecialCharTok{$}\NormalTok{Price), }\AttributeTok{col=}\StringTok{"red"}\NormalTok{)}
\end{Highlighting}
\end{Shaded}

\includegraphics{Regression_files/figure-latex/unnamed-chunk-17-1.pdf}

This tells us that most short flights have 0 stops and are less than
\$10000. It also tells us that the number of stops doesn't necessarily
affect the price, rather just the duration.

\hypertarget{linear-regression}{%
\subsection{Linear Regression}\label{linear-regression}}

\hypertarget{model-for-predictors-duration-total_stops}{%
\subsubsection{Model for predictors Duration +
Total\_Stops}\label{model-for-predictors-duration-total_stops}}

\begin{Shaded}
\begin{Highlighting}[]
\NormalTok{lm1 }\OtherTok{\textless{}{-}} \FunctionTok{lm}\NormalTok{(Price }\SpecialCharTok{\textasciitilde{}}\NormalTok{ Duration }\SpecialCharTok{+}\NormalTok{ Total\_Stops, }\AttributeTok{data=}\NormalTok{train)}
\FunctionTok{summary}\NormalTok{(lm1)}
\end{Highlighting}
\end{Shaded}

\begin{verbatim}
## 
## Call:
## lm(formula = Price ~ Duration + Total_Stops, data = train)
## 
## Residuals:
##    Min     1Q Median     3Q    Max 
##  -9556  -2133   -784   1554  53073 
## 
## Coefficients:
##             Estimate Std. Error t value Pr(>|t|)    
## (Intercept) 5507.860     63.563   86.65   <2e-16 ***
## Duration      77.081      6.771   11.38   <2e-16 ***
## Total_Stops 3383.876     85.180   39.73   <2e-16 ***
## ---
## Signif. codes:  0 '***' 0.001 '**' 0.01 '*' 0.05 '.' 0.1 ' ' 1
## 
## Residual standard error: 3574 on 8542 degrees of freedom
## Multiple R-squared:  0.3822, Adjusted R-squared:  0.382 
## F-statistic:  2642 on 2 and 8542 DF,  p-value: < 2.2e-16
\end{verbatim}

In this Linear Regression model we see the effect that the duration of
the flight and number of stops have on the ticket price. Our R-squared
value is around 0.38, which indicates that the duration and number of
stops are not the best predictors for this model. They don't have as big
of an effect on the ticket price as I thought. This can be because there
are other factors that affect the flight, like the airline, since some
are more luxurious and therefore more expensive, or the amount of days
the ticket was purchased prior to the flight. If tickets are bought a
couple of days before the flight, they are most likely more expensive
than tickets that were purchased ahead in time. The fact that the
p-value is less than 0.5 shows that this model is statistically
significant.

\begin{Shaded}
\begin{Highlighting}[]
\FunctionTok{plot}\NormalTok{(lm1)}
\end{Highlighting}
\end{Shaded}

\includegraphics{Regression_files/figure-latex/unnamed-chunk-19-1.pdf}
\includegraphics{Regression_files/figure-latex/unnamed-chunk-19-2.pdf}
\includegraphics{Regression_files/figure-latex/unnamed-chunk-19-3.pdf}
\includegraphics{Regression_files/figure-latex/unnamed-chunk-19-4.pdf}

Residuals vs Fitted: This represents the difference between the actual
price of the plane ticket and our models prediction. Our model suggests
some form of heteroscedasticity, meaning the variability of our
predictions are not equally variable throughout. Some values are closer
to the 0 line than others. Most values are following a linear
relationship though. So there is a clear relationship between the
predictors and the predicted value.

Normal Q-Q: Most data is in between -2 and 2 standard deviations, just
like in a normal distributions. There is a bunch of extremes on the
right side, which means it is very hard to predict what the price is
going to be based off this model.

Scale-Location: This plot shows if the residuals are spread equally
among our predictions in order to check homoscedasticity. Since there
are clear trends in the plot, there is no equal variance in our
residuals.

Residuals vs Leverage: This plot helps us find influential data points.
We don't have any data points that have large leverage and also high
residuals, meaning, there are few data points that have a big impact on
the coefficients and the intercept of the model. So, we don't have any
data points that we should remove necessarily.

\hypertarget{evaluate-on-the-test-set-for-predictors-duration-and-total_stops}{%
\subsubsection{Evaluate on the test set for predictors Duration and
Total\_Stops}\label{evaluate-on-the-test-set-for-predictors-duration-and-total_stops}}

\begin{Shaded}
\begin{Highlighting}[]
\NormalTok{pred1 }\OtherTok{\textless{}{-}} \FunctionTok{predict}\NormalTok{(lm1, }\AttributeTok{newdata =}\NormalTok{ test)}
\NormalTok{cor1 }\OtherTok{\textless{}{-}} \FunctionTok{cor}\NormalTok{(pred1, test}\SpecialCharTok{$}\NormalTok{Price)}
\NormalTok{mse1 }\OtherTok{\textless{}{-}} \FunctionTok{mean}\NormalTok{((pred1 }\SpecialCharTok{{-}}\NormalTok{ test}\SpecialCharTok{$}\NormalTok{Price) }\SpecialCharTok{\^{}}\DecValTok{2}\NormalTok{)}
\NormalTok{rmse1 }\OtherTok{\textless{}{-}} \FunctionTok{sqrt}\NormalTok{(mse1)}

\FunctionTok{print}\NormalTok{(}\FunctionTok{paste}\NormalTok{(}\StringTok{"correlation:"}\NormalTok{, cor1))}
\end{Highlighting}
\end{Shaded}

\begin{verbatim}
## [1] "correlation: 0.584158942107584"
\end{verbatim}

\begin{Shaded}
\begin{Highlighting}[]
\FunctionTok{print}\NormalTok{(}\FunctionTok{paste}\NormalTok{(}\StringTok{"mse:"}\NormalTok{, mse1))}
\end{Highlighting}
\end{Shaded}

\begin{verbatim}
## [1] "mse: 15580681.0188734"
\end{verbatim}

\begin{Shaded}
\begin{Highlighting}[]
\FunctionTok{print}\NormalTok{(}\FunctionTok{paste}\NormalTok{(}\StringTok{"rmse:"}\NormalTok{, rmse1))}
\end{Highlighting}
\end{Shaded}

\begin{verbatim}
## [1] "rmse: 3947.23713740046"
\end{verbatim}

\hypertarget{model-with-all-predictors}{%
\subsubsection{Model with all
predictors}\label{model-with-all-predictors}}

\begin{Shaded}
\begin{Highlighting}[]
\NormalTok{lma }\OtherTok{\textless{}{-}} \FunctionTok{lm}\NormalTok{(Price }\SpecialCharTok{\textasciitilde{}}\NormalTok{., }\AttributeTok{data =}\NormalTok{ train)}
\FunctionTok{summary}\NormalTok{(lma)}
\end{Highlighting}
\end{Shaded}

\begin{verbatim}
## 
## Call:
## lm(formula = Price ~ ., data = train)
## 
## Residuals:
##    Min     1Q Median     3Q    Max 
##  -7932  -1407   -366   1157  40849 
## 
## Coefficients: (4 not defined because of singularities)
##                                           Estimate Std. Error t value Pr(>|t|)
## (Intercept)                               6671.299    225.572  29.575  < 2e-16
## AirlineAir India                          1728.730    206.412   8.375  < 2e-16
## AirlineGoAir                                31.473    292.967   0.107  0.91445
## AirlineIndiGo                              327.285    197.325   1.659  0.09723
## AirlineJet Airways                        4316.456    194.800  22.158  < 2e-16
## AirlineJet Airways Business              44491.881   1303.749  34.126  < 2e-16
## AirlineMultiple carriers                  3727.769    212.660  17.529  < 2e-16
## AirlineMultiple carriers Premium economy  4405.548    853.178   5.164 2.48e-07
## AirlineSpiceJet                           -305.207    217.159  -1.405  0.15992
## AirlineTrujet                            -1400.622   2888.854  -0.485  0.62780
## AirlineVistara                            2245.938    237.025   9.476  < 2e-16
## AirlineVistara Premium economy            4067.229   1673.804   2.430  0.01512
## SourceChennai                            -2508.810    205.218 -12.225  < 2e-16
## SourceDelhi                              -2532.786    124.337 -20.370  < 2e-16
## SourceKolkata                            -2483.059    122.905 -20.203  < 2e-16
## SourceMumbai                             -4037.828    167.056 -24.171  < 2e-16
## DestinationCochin                               NA         NA      NA       NA
## DestinationDelhi                         -3475.103    147.747 -23.521  < 2e-16
## DestinationHyderabad                            NA         NA      NA       NA
## DestinationKolkata                              NA         NA      NA       NA
## DestinationNew Delhi                            NA         NA      NA       NA
## Dep_Time                                    17.510      5.555   3.152  0.00163
## Duration                                     9.995      5.908   1.692  0.09070
## Total_Stops                               2649.529     82.742  32.022  < 2e-16
##                                             
## (Intercept)                              ***
## AirlineAir India                         ***
## AirlineGoAir                                
## AirlineIndiGo                            .  
## AirlineJet Airways                       ***
## AirlineJet Airways Business              ***
## AirlineMultiple carriers                 ***
## AirlineMultiple carriers Premium economy ***
## AirlineSpiceJet                             
## AirlineTrujet                               
## AirlineVistara                           ***
## AirlineVistara Premium economy           *  
## SourceChennai                            ***
## SourceDelhi                              ***
## SourceKolkata                            ***
## SourceMumbai                             ***
## DestinationCochin                           
## DestinationDelhi                         ***
## DestinationHyderabad                        
## DestinationKolkata                          
## DestinationNew Delhi                        
## Dep_Time                                 ** 
## Duration                                 .  
## Total_Stops                              ***
## ---
## Signif. codes:  0 '***' 0.001 '**' 0.01 '*' 0.05 '.' 0.1 ' ' 1
## 
## Residual standard error: 2879 on 8525 degrees of freedom
## Multiple R-squared:  0.5998, Adjusted R-squared:  0.5989 
## F-statistic: 672.4 on 19 and 8525 DF,  p-value: < 2.2e-16
\end{verbatim}

Compared to our model with just Duration and Total\_Stops as predictors,
this model shows a higher R-squared value, indicating that there are
more factors that influence the price of the plane ticket. As predicted
above, the airline carrier makes a difference, since some tend to me
more expensive.

\hypertarget{evaluate-model-with-all-predictors}{%
\subsubsection{Evaluate model with all
predictors}\label{evaluate-model-with-all-predictors}}

\begin{Shaded}
\begin{Highlighting}[]
\NormalTok{preda }\OtherTok{\textless{}{-}} \FunctionTok{predict}\NormalTok{(lma, }\AttributeTok{newdata =}\NormalTok{ test)}
\NormalTok{cora }\OtherTok{\textless{}{-}} \FunctionTok{cor}\NormalTok{(preda, test}\SpecialCharTok{$}\NormalTok{Price)}
\NormalTok{msea }\OtherTok{\textless{}{-}} \FunctionTok{mean}\NormalTok{((preda }\SpecialCharTok{{-}}\NormalTok{ test}\SpecialCharTok{$}\NormalTok{Price) }\SpecialCharTok{\^{}}\DecValTok{2}\NormalTok{)}
\FunctionTok{print}\NormalTok{(}\FunctionTok{paste}\NormalTok{(}\StringTok{"cor="}\NormalTok{, cora))}
\end{Highlighting}
\end{Shaded}

\begin{verbatim}
## [1] "cor= 0.76968523775651"
\end{verbatim}

\begin{Shaded}
\begin{Highlighting}[]
\FunctionTok{print}\NormalTok{(}\FunctionTok{paste}\NormalTok{(}\StringTok{"mse="}\NormalTok{, msea))}
\end{Highlighting}
\end{Shaded}

\begin{verbatim}
## [1] "mse= 9698939.7919547"
\end{verbatim}

We see pretty decent results. Let's see how they compare to kNN.

\hypertarget{knn-for-regression}{%
\subsection{kNN for regression}\label{knn-for-regression}}

\begin{Shaded}
\begin{Highlighting}[]
\FunctionTok{library}\NormalTok{(caret)}
\end{Highlighting}
\end{Shaded}

\begin{verbatim}
## Loading required package: ggplot2
\end{verbatim}

\begin{verbatim}
## Loading required package: lattice
\end{verbatim}

\begin{Shaded}
\begin{Highlighting}[]
\CommentTok{\# fit the model}
\NormalTok{fit }\OtherTok{\textless{}{-}} \FunctionTok{knnreg}\NormalTok{(train[,}\DecValTok{4}\SpecialCharTok{:}\DecValTok{6}\NormalTok{],train[,}\DecValTok{7}\NormalTok{],}\AttributeTok{k=}\DecValTok{3}\NormalTok{)}

\CommentTok{\# evaluate}
\NormalTok{pred2 }\OtherTok{\textless{}{-}} \FunctionTok{predict}\NormalTok{(fit, test[,}\DecValTok{4}\SpecialCharTok{:}\DecValTok{6}\NormalTok{])}
\NormalTok{cor\_knn1 }\OtherTok{\textless{}{-}} \FunctionTok{cor}\NormalTok{(pred2, test}\SpecialCharTok{$}\NormalTok{Price)}
\NormalTok{mse\_knn1 }\OtherTok{\textless{}{-}} \FunctionTok{mean}\NormalTok{((pred2 }\SpecialCharTok{{-}}\NormalTok{ test}\SpecialCharTok{$}\NormalTok{Price) }\SpecialCharTok{\^{}}\DecValTok{2}\NormalTok{)}
\FunctionTok{print}\NormalTok{(}\FunctionTok{paste}\NormalTok{(}\StringTok{"cor="}\NormalTok{, cor\_knn1))}
\end{Highlighting}
\end{Shaded}

\begin{verbatim}
## [1] "cor= 0.679339454132071"
\end{verbatim}

\begin{Shaded}
\begin{Highlighting}[]
\FunctionTok{print}\NormalTok{(}\FunctionTok{paste}\NormalTok{(}\StringTok{"mse="}\NormalTok{, mse\_knn1))}
\end{Highlighting}
\end{Shaded}

\begin{verbatim}
## [1] "mse= 12759958.583699"
\end{verbatim}

As we can see, the results for kNN weren't quite as good as the results
for the Linear Regression model. (Cor for LinReg: \textasciitilde0.77,
kNN: \textasciitilde0.68) A reason for this difference might be that we
didn't scale the data for kNN, which works better on scaled data.

\hypertarget{scale-the-data-for-knn}{%
\subsubsection{Scale the data for kNN}\label{scale-the-data-for-knn}}

We are scaling both train and test data on the means and standard
deviations of the training set. This is so that information about the
test data does not leak into the scaling.

\begin{Shaded}
\begin{Highlighting}[]
\NormalTok{train\_scaled }\OtherTok{\textless{}{-}}\NormalTok{ train[}\DecValTok{4}\SpecialCharTok{:}\DecValTok{6}\NormalTok{]}
\NormalTok{means }\OtherTok{\textless{}{-}} \FunctionTok{sapply}\NormalTok{(train\_scaled, mean)}
\NormalTok{stdvs }\OtherTok{\textless{}{-}} \FunctionTok{sapply}\NormalTok{(train\_scaled, sd)}
\NormalTok{train\_scaled }\OtherTok{\textless{}{-}} \FunctionTok{scale}\NormalTok{(train\_scaled, }\AttributeTok{center=}\NormalTok{means, }\AttributeTok{scale=}\NormalTok{stdvs)}
\NormalTok{test\_scaled }\OtherTok{\textless{}{-}} \FunctionTok{scale}\NormalTok{(test[, }\DecValTok{4}\SpecialCharTok{:}\DecValTok{6}\NormalTok{], }\AttributeTok{center=}\NormalTok{means, }\AttributeTok{scale=}\NormalTok{stdvs)}
\end{Highlighting}
\end{Shaded}

\hypertarget{knn-on-scaled-data}{%
\subsubsection{kNN on scaled data}\label{knn-on-scaled-data}}

\begin{Shaded}
\begin{Highlighting}[]
\NormalTok{fit }\OtherTok{\textless{}{-}} \FunctionTok{knnreg}\NormalTok{(train\_scaled, train}\SpecialCharTok{$}\NormalTok{Price, }\AttributeTok{k=}\DecValTok{3}\NormalTok{)}
\NormalTok{pred3 }\OtherTok{\textless{}{-}} \FunctionTok{predict}\NormalTok{(fit, test\_scaled)}
\NormalTok{cor\_knn2 }\OtherTok{\textless{}{-}} \FunctionTok{cor}\NormalTok{(pred3, test}\SpecialCharTok{$}\NormalTok{Price)}
\NormalTok{mse\_knn2 }\OtherTok{\textless{}{-}} \FunctionTok{mean}\NormalTok{((pred3 }\SpecialCharTok{{-}}\NormalTok{ test}\SpecialCharTok{$}\NormalTok{Price) }\SpecialCharTok{\^{}}\DecValTok{2}\NormalTok{)}
\FunctionTok{print}\NormalTok{(}\FunctionTok{paste}\NormalTok{(}\StringTok{"cor="}\NormalTok{, cor\_knn2))}
\end{Highlighting}
\end{Shaded}

\begin{verbatim}
## [1] "cor= 0.680064162657665"
\end{verbatim}

\begin{Shaded}
\begin{Highlighting}[]
\FunctionTok{print}\NormalTok{(}\FunctionTok{paste}\NormalTok{(}\StringTok{"mse="}\NormalTok{, mse\_knn2))}
\end{Highlighting}
\end{Shaded}

\begin{verbatim}
## [1] "mse= 12739665.9924593"
\end{verbatim}

The kNN now has a \emph{slightly} higher cor and lower mse than before
scaling, but still not higher than the Linear Regression. This might
just be because we can only use 3 predictor values, because the other
columns contain characters and not numeric values. Compared to the first
Linear Regression model we made, though, using only Duration and
Total\_Stops as predictors, we have a high increase in Correlation and
decrease in mse. (Cor for first lm model: 0.58, for kNN: 0.68)

\hypertarget{find-the-best-k}{%
\subsubsection{Find the best k}\label{find-the-best-k}}

We will try various values of k and plot the results.

\begin{Shaded}
\begin{Highlighting}[]
\NormalTok{cor\_k }\OtherTok{\textless{}{-}} \FunctionTok{rep}\NormalTok{(}\DecValTok{0}\NormalTok{, }\DecValTok{20}\NormalTok{)}
\NormalTok{mse\_k }\OtherTok{\textless{}{-}} \FunctionTok{rep}\NormalTok{(}\DecValTok{0}\NormalTok{, }\DecValTok{20}\NormalTok{)}
\NormalTok{i }\OtherTok{\textless{}{-}} \DecValTok{1}
\ControlFlowTok{for}\NormalTok{ (k }\ControlFlowTok{in} \FunctionTok{seq}\NormalTok{(}\DecValTok{1}\NormalTok{, }\DecValTok{39}\NormalTok{, }\DecValTok{2}\NormalTok{)) \{}
\NormalTok{  fit\_k }\OtherTok{\textless{}{-}} \FunctionTok{knnreg}\NormalTok{(train\_scaled, train}\SpecialCharTok{$}\NormalTok{Price, }\AttributeTok{k=}\NormalTok{k)}
\NormalTok{  pred\_k }\OtherTok{\textless{}{-}} \FunctionTok{predict}\NormalTok{(fit\_k, test\_scaled)}
\NormalTok{  cor\_k[i] }\OtherTok{\textless{}{-}} \FunctionTok{cor}\NormalTok{(pred\_k, test}\SpecialCharTok{$}\NormalTok{Price)}
\NormalTok{  mse\_k[i] }\OtherTok{\textless{}{-}} \FunctionTok{mean}\NormalTok{((pred\_k }\SpecialCharTok{{-}}\NormalTok{ test}\SpecialCharTok{$}\NormalTok{Price)}\SpecialCharTok{\^{}}\DecValTok{2}\NormalTok{)}
  \FunctionTok{print}\NormalTok{(}\FunctionTok{paste}\NormalTok{(}\StringTok{"k="}\NormalTok{, k, cor\_k[i], mse\_k[i]))}
\NormalTok{  i }\OtherTok{\textless{}{-}}\NormalTok{ i }\SpecialCharTok{+} \DecValTok{1}
\NormalTok{\}}
\end{Highlighting}
\end{Shaded}

\begin{verbatim}
## [1] "k= 1 0.679532385149251 12775379.3365894"
## [1] "k= 3 0.680064162657665 12739665.9924593"
## [1] "k= 5 0.684164787758733 12582162.1263527"
## [1] "k= 7 0.683470613982923 12602502.1739109"
## [1] "k= 9 0.680956410633889 12683507.5976628"
## [1] "k= 11 0.671786051022922 12976684.7561946"
## [1] "k= 13 0.667703952965126 13105256.5607836"
## [1] "k= 15 0.670920631314766 13003740.4050598"
## [1] "k= 17 0.67083363096341 13008086.8167307"
## [1] "k= 19 0.668188392789619 13093213.7791912"
## [1] "k= 21 0.66732212881467 13120877.4467431"
## [1] "k= 23 0.666704476200499 13140518.5321617"
## [1] "k= 25 0.664163641127559 13219073.0067556"
## [1] "k= 27 0.662103413313666 13282726.2014171"
## [1] "k= 29 0.661556475971501 13301342.8765166"
## [1] "k= 31 0.662513906760468 13273991.243621"
## [1] "k= 33 0.661163394757784 13318670.9156331"
## [1] "k= 35 0.661938828399245 13296059.8344066"
## [1] "k= 37 0.661897464064353 13299176.0080797"
## [1] "k= 39 0.654935199111982 13508553.9760139"
\end{verbatim}

\begin{Shaded}
\begin{Highlighting}[]
\FunctionTok{plot}\NormalTok{(}\DecValTok{1}\SpecialCharTok{:}\DecValTok{20}\NormalTok{, cor\_k, }\AttributeTok{lwd=}\DecValTok{2}\NormalTok{, }\AttributeTok{col=}\StringTok{\textquotesingle{}red\textquotesingle{}}\NormalTok{, }\AttributeTok{ylab=}\StringTok{""}\NormalTok{, }\AttributeTok{yaxt=}\StringTok{\textquotesingle{}n\textquotesingle{}}\NormalTok{)}
\FunctionTok{par}\NormalTok{(}\AttributeTok{new=}\ConstantTok{TRUE}\NormalTok{)}
\FunctionTok{plot}\NormalTok{(}\DecValTok{1}\SpecialCharTok{:}\DecValTok{20}\NormalTok{, mse\_k, }\AttributeTok{lwd=}\DecValTok{2}\NormalTok{, }\AttributeTok{col=}\StringTok{\textquotesingle{}blue\textquotesingle{}}\NormalTok{, }\AttributeTok{labels=}\ConstantTok{FALSE}\NormalTok{, }\AttributeTok{ylab=}\StringTok{""}\NormalTok{, }\AttributeTok{yaxt=}\StringTok{\textquotesingle{}n\textquotesingle{}}\NormalTok{)}
\end{Highlighting}
\end{Shaded}

\begin{verbatim}
## Warning in plot.window(...): "labels" is not a graphical parameter
\end{verbatim}

\begin{verbatim}
## Warning in plot.xy(xy, type, ...): "labels" is not a graphical parameter
\end{verbatim}

\begin{verbatim}
## Warning in box(...): "labels" is not a graphical parameter
\end{verbatim}

\begin{verbatim}
## Warning in title(...): "labels" is not a graphical parameter
\end{verbatim}

\includegraphics{Regression_files/figure-latex/unnamed-chunk-32-1.pdf}

As we can see, the best value for k is at k = 5 (in this plot it is at
3, but that is because there is k=5 is at index 3 in lists cor\_k and
mse\_k)

We can also check with min and max:

\begin{Shaded}
\begin{Highlighting}[]
\FunctionTok{which.min}\NormalTok{(mse\_k)}
\end{Highlighting}
\end{Shaded}

\begin{verbatim}
## [1] 3
\end{verbatim}

\begin{Shaded}
\begin{Highlighting}[]
\FunctionTok{which.max}\NormalTok{(cor\_k)}
\end{Highlighting}
\end{Shaded}

\begin{verbatim}
## [1] 3
\end{verbatim}

Since we have used k=3 (which was just a coincidence) in the above kNN
regression already, we have our best data for this regression.

\hypertarget{decision-tree-for-regression}{%
\subsection{Decision Tree for
regression}\label{decision-tree-for-regression}}

\begin{Shaded}
\begin{Highlighting}[]
\FunctionTok{library}\NormalTok{(tree)}
\FunctionTok{library}\NormalTok{(MASS)}
\NormalTok{tree1 }\OtherTok{\textless{}{-}} \FunctionTok{tree}\NormalTok{(Price }\SpecialCharTok{\textasciitilde{}}\NormalTok{., }\AttributeTok{data=}\NormalTok{train)}
\FunctionTok{summary}\NormalTok{(tree1)}
\end{Highlighting}
\end{Shaded}

\begin{verbatim}
## 
## Regression tree:
## tree(formula = Price ~ ., data = train)
## Variables actually used in tree construction:
## [1] "Duration"    "Total_Stops"
## Number of terminal nodes:  3 
## Residual mean deviance:  11910000 = 1.017e+11 / 8542 
## Distribution of residuals:
##    Min. 1st Qu.  Median    Mean 3rd Qu.    Max. 
## -8104.0 -1996.0  -393.2     0.0  1783.0 52040.0
\end{verbatim}

\begin{Shaded}
\begin{Highlighting}[]
\NormalTok{pred4 }\OtherTok{\textless{}{-}} \FunctionTok{predict}\NormalTok{(tree1, }\AttributeTok{newdata=}\NormalTok{test)}
\NormalTok{cor\_tree }\OtherTok{\textless{}{-}} \FunctionTok{cor}\NormalTok{(pred4, test}\SpecialCharTok{$}\NormalTok{Price)}
\FunctionTok{print}\NormalTok{(}\FunctionTok{paste}\NormalTok{(}\StringTok{"cor:"}\NormalTok{, cor\_tree))}
\end{Highlighting}
\end{Shaded}

\begin{verbatim}
## [1] "cor: 0.63007524717813"
\end{verbatim}

\begin{Shaded}
\begin{Highlighting}[]
\NormalTok{mse\_tree }\OtherTok{\textless{}{-}} \FunctionTok{mean}\NormalTok{((pred4 }\SpecialCharTok{{-}}\NormalTok{ test}\SpecialCharTok{$}\NormalTok{Price) }\SpecialCharTok{\^{}}\DecValTok{2}\NormalTok{)}
\FunctionTok{print}\NormalTok{(}\FunctionTok{paste}\NormalTok{(}\StringTok{"mse:"}\NormalTok{, mse\_tree))}
\end{Highlighting}
\end{Shaded}

\begin{verbatim}
## [1] "mse: 14271349.8899186"
\end{verbatim}

So far, the correlation and mse are worse than for either kNN or Linear
Regression.

\begin{Shaded}
\begin{Highlighting}[]
\FunctionTok{plot}\NormalTok{(tree1)}
\FunctionTok{text}\NormalTok{(tree1, }\AttributeTok{cex =} \FloatTok{0.5}\NormalTok{, }\AttributeTok{pretty =} \DecValTok{0}\NormalTok{)}
\end{Highlighting}
\end{Shaded}

\includegraphics{Regression_files/figure-latex/unnamed-chunk-38-1.pdf}

\hypertarget{concusion}{%
\subsection{Concusion}\label{concusion}}

Comparing all the results, kNN and Linear Regression performed the best.
In the dataset there are only 3 predictors that are numeric values,
which made it hard to fairly compare all 3 models. The Linear Regression
model got the best results by using all predictors. The kNN got the best
results considering it could only use 3 predictors. And even then, it
was only slightly worse than the Linear Regression model using all
predictors. The Decision tree performed slightly worse than the kNN, as
it was also only using 2 predictors. For this dataset specifically, the
Linear Regression was the most powerful model. I think these results
strongly differ with different types of data sets.

\end{document}
